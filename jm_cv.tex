%%%%%%%%%%%%%%%%%%%%%%%%%%%%%%%%%%%%%%%%%
% Medium Length Professional CV
% LaTeX Template
% Version 2.0 (8/5/13)
%
% This template has been downloaded from:
% http://www.LaTeXTemplates.com
%
% Original author:
% Rishi Shah 
%
% Important note:
% This template requires the resume.cls file to be in the same directory as the
% .tex file. The resume.cls file provides the resume style used for structuring the
% document.
%
%%%%%%%%%%%%%%%%%%%%%%%%%%%%%%%%%%%%%%%%%

%----------------------------------------------------------------------------------------
%	PACKAGES AND OTHER DOCUMENT CONFIGURATIONS
%----------------------------------------------------------------------------------------

\documentclass{resume} % Use the custom resume.cls style

\usepackage[left=0.75in,top=0.6in,right=0.75in,bottom=0.6in]{geometry} % Document margins
\newcommand{\tab}[1]{\hspace{.2667\textwidth}\rlap{#1}}
\newcommand{\itab}[1]{\hspace{0em}\rlap{#1}}
\usepackage{hyperref}
\name{Shihang Hou} % Your name

\begin{document}
\begin{rSection}{Contact}
\textbf{Address}:
St Cross College \hfill \\ 
61 St Giles' \\
Oxford OX1 3LZ \\
\textbf{Tel}: +44 7597 509060 \\
\textbf{Email}: \href{mailto:shihang.hou@economics.ox.ac.uk}{shihang.hou@economics.ox.ac.uk} \\
\textbf{Personal Site}: \href{https://www.shihanghou.com}{https://www.shihanghou.com} \\
\textbf{Nationality:} Singaporean

\end{rSection}

\begin{rSection}{References}
\begin{tabular}{lr}
% Referee 1
\begin{minipage}[t]{3in}
\textbf{Prof.\ Margaret Stevens}\\
University of Oxford \\
\href{mailto:margaret.stevens@economics.ox.ac.uk}{margaret.stevens@economics.ox.ac.uk}
\end{minipage}
&
% Referee 2
\begin{minipage}[t]{3in}
\textbf{Prof.\ Barbara Petrongolo}\\
University of Oxford \\
\href{mailto:barbara.petrongolo@economics.ox.ac.uk}{barbara.petrongolo@economics.ox.ac.uk}
\end{minipage}
\\
\\ % Additional newline for spacing.
% Referee 3
\begin{minipage}[t]{3in}
\textbf{Prof.\ Abigail Adams-Prassl}\\
University of Oxford \\
\href{mailto:abi.adams-prassl@economics.ox.ac.uk}{abi.adams-prassl@economics.ox.ac.uk}
\end{minipage}
\end{tabular}
 
 
\end{rSection}


\begin{rSection}{Education}

{\bf University of Oxford} \\
DPhil in Economics \hfill {\em 2018 - 2023 (expected)} 
\\ \textit{``Essays on the Relation between Labour Market Outcomes and Higher Education"}
\\ Supervised by Professor Margaret Stevens, and Professor Abigail Adams-Prassl

MPhil in Economics \hfill {\em 2016 - 2018}
\\ \textit{``The Role of Expected Wages in Choosing a Subject of Study in Higher Education"}, supervised by Professor Margaret Stevens

BA in Philosophy, Politics and Economics \hfill {\em 2013 - 2016}
\\ St Hilda's College, specialising in Philosophy and Economics

\end{rSection}

\begin{rSection}{Teaching and Research Fields}
Labour economics; Economics of education
\end{rSection}

\begin{rSection}{Job Market Paper}
{\bf The Role of Uncertainty in Education Mismatch} \\ (Job Market Paper)

In many developed economies, there is both a high share of students selecting into higher education (HE) and a large share of graduates who end up working in non-graduate occupations. Does this mean that the level of education is inefficiently high? In this paper, I propose an explanation of overeducation based on the observation that the causal effect of HE on human capital is both heterogeneous and imperfectly observed by the student. I propose a model in which workers first decide whether to invest in higher education on the basis of limited information, and then match to jobs in a labour market. In this setting, workers could be over-educated if they do not benefit from HE after attending, and under-educated if they could have benefited but did not attend HE. Structurally estimating my model on UK data, I calculate that 18.2\% of the population are over-educated and 14.7\% are under-educated. In general, the equilibrium in the model is inefficient, because of a hold-up externality which causes under-investment and a congestion externality which causes over-investment. These externalities offset each other, and in general, it is ambiguous whether the level of education is inefficiently high or low. 
\end{rSection}

\begin{rSection}{Working Papers}
{\bf The Butcher, the Brewer, or the Baker: The Role of Occupations in Explaining Wage Inequality} \\ (with Luke Heath Milsom)

Google does not pay its canteen chefs the same premium as it does its software engineers; there is heterogeneity in pay premia within a firm. To study the implications of this observation, we estimate a two-way, worker-job, fixed effect model that builds upon the canonical model by Abowd et al. (1999) by allowing for within-firm heterogeneity. Estimating our model on UK administrative data, we find that even if workers, and firms, were identical, 25\% of current log-wage variance would remain, due solely to heterogeneity between occupations. We further document that worker heterogeneity accounts for a relatively small proportion of log-wage variance, and that between-firm pay heterogeneity is more important in higher-wage occupations and larger labour markets.

\vspace{1mm}

{\bf Determinants of University Subject Choice in the UK}

Students applying to British universities have to choose to specialise in a particular subject at the point of application. What factors play into their choice of subject of study? I analyse students' subject choices in university using a cohort panel dataset (LSYPE), and in particular, whether those choices respond to their expected wages conditional on studying the subject. I find that earnings, defined as earnings at age 25 or as lifetime earnings, are positively associated with choice probability, although the quantitative impact of expected earnings on subject choice is small. I also find that there are substantial differences in choice probability between students of different sexes, ethnicities and to a lesser extent, socio-economic classes.

\end{rSection}

\begin{rSection}{Works-in-Progress}
 
{\bf Labours of Love: Search for Occupational Meaning and Worker Welfare} \\ (with Luke Heath Milsom and Hannah Zillessen)

{\bf How Informative are Grades about Labour Market Outcomes?}

\end{rSection}

\begin{rSection}{Research Experience}

{\bf With Prof Margaret Stevens and Prof Patricia Rice} \hfill Jun 2018 - Jan 2020 \\
Assisted on a project studying differences in achievement by students of different ethnicities at the University of Oxford.

{\bf With Prof Imogen Goold} \hfill Dec 2018 – Mar 2020 \\
Assisted in reviewing admissions procedures at the Law Faculty for undergraduates by analysing admissions data and making recommendations for changes in the admission system.

\end{rSection}

\begin{rSection}{Teaching Experience}
 
\begin{itemize}
    \itemsep0em
    \item TA in Macroeconomics for the MSc in Economics for Development (2018-19), under Prof Christopher Adam. 
    \item Non-stipendiary Lecturer, FHS Microeconomics and Quantitative Economics, Christ Church (2019-20), under Prof Petr Sedlacek.
    \item Stipendiary Lecturer, FHS Microeconomics and Quantitative Economics, New College (2020-21), under Dr Richard Mash.
    \item TA in Quantitative Economics for the MSc in Economics for Development (2021-22), under Prof Christopher Woodruff. 
    \item TA in Stata for the MSc in Economics for Development (2022-23), under Prof Christopher Adam.
\end{itemize} 

\end{rSection}

\begin{rSection}{Conferences and External Presentations}
 \textbf{2022:} Royal Economic Society Annual Conference; Warwick 10th PhD Conference; University of Surrey Micro PhD Workshop; Young Economist Symposium; Asian \& Australasian Society of Labour Economics Conference 2022* \\
 \textbf{2021:} Warwick 9th PhD Conference (presented by co-author) \\
$^*$ scheduled
\end{rSection}

\begin{rSection}{Other Professional Activities}
\begin{itemize}
    \item Interviewer in Economics and Management undergraduate admissions (Exeter College, 2018; Christ Church, 2019)
    \item Organiser of Applied Microeconomics group DPhil reading group (2018-20)
    \item Co-organiser of DPhil Structural Estimation reading group (2021-22)
\end{itemize}
\end{rSection}

\begin{rSection}{Skills and Languages}
 \textbf{Languages:} Mandarin Chinese (native), English (fluent) \\
 \textbf{Software:} Stata, R, Matlab, Julia, OxMetrics, LaTeX, SQL, Microsoft Office suite
\end{rSection}

\end{document}
